
\usepackage{ifxetex}                      %% Для сборки документа и pdflatex'ом, и xelatex'ом
\ifxetex
    %% xelatex
    \usepackage{polyglossia}                       %% загружает пакет многоязыковой вёрстки
    \setdefaultlanguage[spelling=modern]{russian}  %% устанавливает главный язык документа
    \setotherlanguage{english}                     %% объявляет второй язык документа
    \defaultfontfeatures{Ligatures={TeX}}          %% свойства шрифтов по умолчанию
    \setmainfont[Ligatures={TeX}]{Old Standard}    %% задаёт основной шрифт документа
    \setsansfont{Old Standard}                     %% задаёт шрифт без засечек (но Old Standard с засечками заменить на DejaVu Sans?)
    \setmonofont{DejaVu Sans Mono}                  %% задаёт моноширинный шрифт
\else
    %% pdflatex
    \usepackage{cmap}                     %% Поиск русских  слов в pdf
    \usepackage[T2A]{fontenc}             %% Внутренняя кодировка шрифта
    \usepackage[utf8]{inputenc}           %% Кодировка исходного текста;
    %                                      % new: [utf8] (for XeLaTeX);
    %                                      % unmaintained: [ucs];
    %                                      % old, but work: [utf8x] (load ucs)
    %                                      % koi8-r
    %                                      % можно указать cp866 (Alt-кодировка DOS)
    \usepackage[english,russian]{babel}   %% Поддержка русского текста:
    %                                      % включение русификации, русских и
    %                                      % английских стилей и переносов
\fi  



% \hyphenpenalty=50   % переносы разрешены (значение по умолчанию)
% \hyphenpenalty=10000 % запретить переносы
%\hyphenpenalty=9999

% badness - мера разрежённости строки (badness=0 всё хорошо).
% При разбиении абзаца на строки Latex не может создать такие строки,
% badness которых больше, чем значение параметра \tolerance
%\tolerance=10

% к пределу растяжимости каждой из строк в процессе разбиения абзаца
% на строки и вычислений соответствующих значений badness прибавляется
% значение \emergencystretch. Чем больше значение параметра
% \emergencystretch, тем более разреженные строки появятся на печати.
% например: \emergencystretch=10pt конфигурирует TeX для использования
% не более 10pt дополнительного пробела для каждой строки
%\emergencystretch=10pt

% запрет заездов на правое поле документа и переносы. it tells
% TeX to not even look for hyphenation positions
%\pretolerance=9999

% Only words containing at least 16 characters will now be hyphenated.
%\lefthyphenmin=4  % least 4 characters before the hyphen
%\righthyphenmin=4 % least 4 characters after the hyphen


% запрещение вдов ("widow" - короткая строка или слово в конце абзаца)
\widowpenalty=10000
% запрещение сирот ("orphan" - одиночная строка вылетевшая на следующую страницу или колонку)
\clubpenalty=10000
%
% unknown:
%\brokenpenalty=4991
%\predisplaypenalty=10000
%\postdisplaypenalty=1549
%\displaywidowpenalty=1602

\usepackage{graphics}
\usepackage{wrapfig}
\usepackage{multicol}
\usepackage{multirow}
%\usepackage{fullpage}
\usepackage{textcomp} % типографские значки
\usepackage{mathtext} % если нужны русские буквы в формулах
\usepackage{gensymb}  % для спец знаков
\usepackage{amsmath} % для спец знаков в формулах
\usepackage{amssymb} % для спец знаков в формулах
\usepackage{topcapt} % подписи к таблицам
\usepackage{dcolumn} % выравнивание чисел
\usepackage{ulem} % подчёркивание

\usepackage{makeidx} % индекс

\usepackage{import} % for converted svg to pdf

% вращение 
\usepackage{lscape}     % for %\begin{landscape} ...   %\end{landscape}
\usepackage{rotating}   % for sideways and \rotatebox{-90}{}

% водяные знаки
%\usepackage[firstpage]{draftwatermark}
%\SetWatermarkScale{0.7}
%\SetWatermarkLightness{0.9}
%\SetWatermarkText{\textbf{Рабочая версия}}

\usepackage[iso,english]{isodate}
\usepackage{datetime2} % print datetime in GOST ISO 8601-2001

\frenchspacing

\usepackage{fancyvrb} % for verbatim text

\usepackage{listings} % for source code
%\lstloadlanguages{lisp}
\lstset{
  language=C,
  %basicstyle=\tiny, %or \small or \footnotesize etc.
  extendedchars=\true, % for russian characters in comments
  %texcl,  % for spaces in russian characters in comments
  keepspaces = true, % for spaces in russian characters in comments. it's work!
  escapechar=|,
  frame=single,
  commentstyle=\itshape,
  inputencoding=utf8,
  stringstyle=\bfseries
}

\usepackage{csvsimple}

% fixme: remove before release
\usepackage[colorinlistoftodos,prependcaption,textsize=tiny]{todonotes}
\setlength{\marginparwidth}{3cm}

