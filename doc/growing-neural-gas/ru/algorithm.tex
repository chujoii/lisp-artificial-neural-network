\documentclass[unicode, 12pt, a4paper,oneside,fleqn]{article}

\input{../../../TeX/preambula-ru.tex}
\usepackage[colorlinks=true]{hyperref} % url hyperlink (beamer already include it, so move here for prevent conflict)

\author{Роман В. Приходченко}

\title{Алгоритм растущего нейронного газа}


\makeindex



\begin{document}

% меняем английские термины на русские
\renewcommand\bibname{СПИСОК ЛИТЕРАТУРЫ}
\renewcommand\refname{\centering Список литературы}
\renewcommand\contentsname{\centering Содержание}


% образцы переноса сложных слов - не работает?
% \hyphenation{веб=-ин-тер-фей-се веб-ин-тер-фей-с}
% or use in text: веб"=интерфейс (require: \usepackage[russian]{babel})

% печатаем титульный лист
\makeatletter % generate \@title, \@date, ...
\maketitle

\newpage
% печатаем оглавление
\tableofcontents

\newpage
Описание алгоритма основан на статьях \cite[A "Neural-Gas" Network Learns Topologies]{neural-gas-T.Martinetz-K.Schulten}, \cite[Нейронный газ]{ru.wikipedia.org}, \cite[Растущий нейронный газ - реализация на языке программирования MQL5]{gng-subbotin} приведённых в списке литературы.
\section{}



\newpage
\bibliographystyle{plain}
\bibliography{algorithm}

\end{document}
